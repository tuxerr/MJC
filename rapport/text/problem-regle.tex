% Règles

Deux règles nous ont posés problèmes :\\
\begin{verbatim}
 INST → return E pv 
 INST → INST -> si paro  E  parf BLOC SIX.
\end{verbatim}

Ces règles sont problématiques à cause de la gestion de la règle E → ER AFFX. On
ne doit pas autoriser à priori des instructions telles que :\\
 « return (a = 2) ;» ou « if (a = 2) {...} ».\
On a envisagé deux solutions différentes : la première est de tout simplement
changer la grammaire, en modifiant ces règles pour donner : \\

\begin{verbatim}
INST → return ER pv
INST → INST -> si paro  ER  parf BLOC SIX,
\end{verbatim}

ce qui aurait résolu le problème en renvoyant une erreur syntaxique.\\

La deuxième solution, et celle qui a été finalement implémentée, est de renvoyer
une erreur analytique, calculée lors de l'évaluation de la règle\\

\begin{verbatim}
 E → ER AFFX.
\end{verbatim}

