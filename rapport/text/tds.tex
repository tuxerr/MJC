%%Inclusion du graphe.png qui servira de support aux explications

La TDS que nous avons choisie est une variation de celle qui nous a été donnée
en TP de TDL. En effet, au lieu d'hériter d'une HashMap<String,INFO>, elle
contient en tant qu'attributs 3 HashMaps de variables, méthodes et classes
respectivement, comme il est visible sur le graphe UML. Le principe même d'une
HashMap est d'être une bijection entre la clef de type String et la Valeur de
type INFO. Or,il est possible en MJ de donner le même nom à une variable et une
méthode, et le même nom à un classe qu'à un constructeur. Par son principe même
une seule HashMap ne peut pas gérer cela. Ainsi, nos fonctions
chercherGlobalement existent en trois exemplaires : pour les variables, les
classes et les méthodes.\\

Afin d'autoriser la surcharge de méthode, la fonction chercherGlobalementMethod
prend aussi en argument les arguments auquels la méthode doit corresponde pour être
considérée comme acceptable. La HashMap de méthodes est en effet, pour des soucis
d'optimisation, contient des LMETHODES, une classe de liste de méthodes ayant le
même nom (mais des arguments différents; et il n'y a pas de surcharge du type de
retour). Ainsi, chercherGlobalementMethode(nom,arguments) trouve donc d'abord
dans la HashMap de la TDS la liste de méthodes ayant ce nom, puis trouve dans
cette liste une méthode avec les bons arguments.\
Ce système nous permet de déclarer des constructeurs qui ont le même nom que
leur classe associé, ainsi que de déclarer un attribut ayant le même nom que sa
classe.\\

Nous avons de plus conservé le système de DTYPE; le type POINTEUR étant une
classe fille qui contient également le DTYPE pointé.\\

Un des problèmes que nous avons rencontré est celui du typage de Micro-Java créé
par la distinction entre type réel et type apparent. En effet, il est possible
pour un exemple contenant une classe P ayant comme classe fille PC d'avoir une
définition :

\begin{verbatim}
 
  P p = new PC();

\end{verbatim}

Ici, PC à pour type ``Pointeur pointant vers PC'' et P a pour type ``Pointeur
pointant vers P'', mais l'affectation était alors toujours valide. Nous avons
donc développé une fonction canAccept pour les DTYPE. Elle permet de déterminer
si l'affectation est valide bien que les types ne soient pas égaux. C'est le cas
si le pointur est nul.\
En effet, pour chaque classe, nous avons défini une liste (générée lors de la
2ème passe) contenant de classes susceptibles d'être Super-Classes lors d'une
affectation. Cette fonction est par exemple utilisée dans ARGLIST (les listes
d'argumentsde fonctions) pour autoriser le passage d'un PC en tant qu'argument
dont le type était P.\\

La classe METHODE contient les informations des méthodes et permet de générer
les vtables. Pour chaque méthode Cette classe contient une étiquette,
positionnée au début du code de cette méthode. On peut ainsi générer son entrée
vtable ainsi qu'un numéro qui lui est associé. En effet, chaque méthode de
classe est numérotée de 0 à nb\_methode-1, ce qui permet pour l'appel des
méthodes de savoir dans quelle position elle est dans la vtable ( plus de détail
sur la liasion tardive et les interface partie ??).


