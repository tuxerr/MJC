
On peut rencontrer deux types de surcharge :\\
\begin{itemize}
	\item Surcharge entre variables attributs, méthodes et classe.
	\item Surcharge des méthodes.
\end{itemize}

Concernant la première surcharge l'exemple suivant doit être acceptable :\\

\begin{verbatim}
class x {
  int x ;
  x (int xi) {
    x = xi ;
  }
}
\end{verbatim}

Dans l'exemple précédent, on a : la classe x, l'attribut x et le constructeur
(la méthode x) de cette classe. Il ne peut pas y avoir d’ambiguïté entre ces
trois objets au niveau sémantique. Afin de simplifier les insertions et les
recherches dans la TDS, on a décidé de créer 3 HashMaps différentes dans chaque
TDS, une pour les classes, une pour les méthodes et une pour les variables
locales et attributs, comme on le voit dans la partie TDS.\\

La surcharge de méthode est très utilisée en Java. En particulier la surcharge
de constructeur :\\

\begin{verbatim}
class test {
   int x ;
   int y ;
   test (int xi) {
     x = xi ;
     y = 0 ;
   }
   test (int xi, int yi) {
     x = xi ;
     y = yi ;
   }
}
\end{verbatim}

Dans l'exemple précédent, on doit pouvoir distinguer les deux constructeurs
indépendamment. Ces deux constructeurs possèdent le même nom, et ont donc la
même entrée dans la HashMap des méthodes. On a résolu ce problème en donnant
comme valeur à la clé correspondante à chaque nom de méthode, une liste de
méthode distinguées par le nombre et le type d'arguments.


