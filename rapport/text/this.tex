\begin{verbatim}
F → this pv
\end{verbatim}

Cette règle introduit l'identificateur this qui renvoit l'instanciation de la
classe courante. Cependant, cette règle ne suffit pas à simuler toutes les
fonctionnalités du this en Java, on ne peut notamment pas faire de
this.methode() ou this.attribut . Néanmoins cela ne restreint pas les
possibilités du langage vu qu'il suffit de créer une variable globale contenant
la valeur du this comme ceci :\\

\begin{verbatim}
class Exemple {
  int x ;
  Exemple () {
    x = 0
  }
  int getX() {
    Exemple e = this ;
    return e.x ;
  }
}
\end{verbatim}

La structure globale du .egg permettrait de gérer les règles this.methode() ou
this.attribut sans trop de difficultés, comme dans la règle 
\begin{verbatim}
F → ident Q 
\end{verbatim}
seul le manque de temps nous a empêché
de traiter ces cas.\\


