Fonctionnement de la liaison tardive

Le mécanisme de la liaison tardive dans notre projet est basé sur l'utilisation pointeur de taille 2, le premier pointant vers l'instance de l'objet (attributs plus VTABLE contenant les méthodes), le 2ème pointant vers ce que nous appelons une VTI (vtable d'interface), ce qui nous permet de gérer le pattern java ou une classe, par héritage, peut implémenter de multiples interfaces n'étant pas héritées les unes des autres (voir schéma explicatif du pattern d'héritage multiple). Ce mécanisme est schématisé dans l'image ci-dessus.

Premier pointeur vers l'instance de l'objet:
Le premier pointeur est nécessaire dans tous les cas de pointeurs (contrairement au 2ème, uniquement utile dans le cas ou le type apparent est une interface). Il contient 2 types d'informations : en haut du pointeur (adresses supérieures), les attributs, et en bas du pointeur, en déplacement négatif, les méthodes. Les méthodes sont choisies à l'instantiation de l'objet en parcourant récursivement les TDS de la classe du type apparent pour déterminer quelles méthodes seront accessibles depuis cette instance. En effet, sachant que MicroJava n'accepte pas les casts, les méthodes accessibles durant toute la vie de l'objet seront un subset des méthodes accessibles de la classe apparente. Après la création de cette liste des méthodes accessibles provenant de la classe apparente, notre algorithme parcourt les méthodes de la classe réelle pour déterminer quelles méthodes appeller (si la méthode est redéfinie dans la classe réelle, celle de la classe réelle sera choisie). Les étiquettes des méthodes trouvées sont insérées selon le numéro de la méthode (de 0 à nb\_methode-1) dans la VTABLE située en dessous du premier pointeur (en bleu dans le dessin).
Dans l'ordonnancement de la VTABLE, les premières méthodes sont toujours celles des classes mères. Par exemple, si une classe A définit une méthode getx, et une classe B extends A définit gety, une VTABLE d'un élément de type apparent B contiendra, dans cet ordre, [etiquette\_getx etiquette\_gety]. Ce système d'ordonnancement permet de gérer automatiquement le problème du B b = new B(); A a = b; En effet, même si la vtable de b contiendra des méthodes en trop, cette vtable commencera bien par les mêmes méthodes, ordonnancées de la même façon qu'une vtable de A. On peut donc simplement copier le pointeur sans risque.

Deuxième pointeur vers la VTI:
Le deuxième pointeur, initialement null (0), pointe vers la VTI de l'interface si le type apparent du pointeur est une interface. Cette VTI est le système choisi pour gérer le pattern java permettant à une classe, via héritage, d'implémenter un nombre ``infini'' d'interfaces n'ayant aucune relation d'ordre entre elles. En effet, l'instruction A a = b; de l'explication du premier pointeur fonctionne grâce à la relation d'ordre existant entre les méthodes de A et de B, sachant que B hérite de A. Cette relation d'ordre n'existe pas entre des interfaces, qui ne sont pas forcément héritées les unes des autres.
Nos VTIs sont une table contenant, comme montré dans le schéma, les déplacements relatifs des méthodes de l'interface par rapport à la VTABLE du pointeur. Considérons, pour rester dans l'exemple du schéma, une classe A ayant les méthodes getx, gety, setx et sety (dans cet ordre, 1 2 3 4, dans la VTABLE). Une interface IA, elle, ne contient que gety et sety. Alors, comme dans l'exemple, la VTI de cette interface contiendra [2, 4]. Ceci permettra, lors d'un appel ia.gety(); de savoir la méthode à appeller. En effet, lors du ia.gety(), on remarquera que gety est la première méthode du type apparent, ia. On ira donc chercher le premier élément de la VTI, 2. Ceci sera le numéro de la méthode de la VTABLE à appeller. Celle ci, dans la VTABLE de A, est gety. La bonne méthode sera donc appellée. 
La VTI n'est donc pas générée à l'instantiation de la classe, qui elle ne génère que la VTABLE (premier pointeur), mais à l'affectation d'une valeur (de type apparent classe) à une interface (ib=a); Ce système nous permet de gérer une implémentation d'un nombre infini d'interfaces via le pattern extends/implements java avec gestion de la liaison tardive, comme notre test SuperTestI.mj le prouve. 


