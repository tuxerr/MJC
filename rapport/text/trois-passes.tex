Nous avons choisi dans notre projet d'effectuer 3 passes du code pour résoudre
les problèmes de déclaration en avant dans le code, et aussi d'appel du
constructeur d'une classe pendant la définition de celle ci (par exemple, la
classe Point a une méthode milieu qui instancie un Point avant la déclaration
complète de Point).\\
Les 3 passes ont chacune une utilité différente :\\ 
\begin{itemize}
 \item Dans la première passe, on introduit dans la TDS toutes les classes présentes dans le programme.
 \item Dans la deuxième passe, on introduit dans la TDS tous les attributs et les méthodes présents dans le programme, et on gère également les extends.
 \item Dans la troisième passe on effectue le reste de l'analyse syntaxique, et on génère enfin le code.\\
\end{itemize}

Pour gérer ce système à 3 passes, nous avons programmé 3 fichiers .egg (MJAVA,
MJAVA2 et MJAVA3) et modifié certains fichiers dans le répertoire compiler/. En
effet, nous avons eu à changer les fichiers MJC.java et MJavaSourceFile.java
pour faire passer un argument de type TDS de passe à passe.\\




